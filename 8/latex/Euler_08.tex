\documentclass{article}
\usepackage[margin=0.5in]{geometry}
\setlength\parindent{0pt}

\title{Euler's project problem 8}
\author{John Fox}
\Large
\begin{document}
\maketitle

\section*{Problem statement} 
The four adjacent digits in the 1000-digit number that have the greatest product are 9 × 9 × 8 × 9 = 5832.
73167176531330624919225119674426574742355349194934\\
96983520312774506326239578318016984801869478851843\\
85861560789112949495459501737958331952853208805511\\
12540698747158523863050715693290963295227443043557\\
66896648950445244523161731856403098711121722383113\\
62229893423380308135336276614282806444486645238749\\
30358907296290491560440772390713810515859307960866\\
70172427121883998797908792274921901699720888093776\\
65727333001053367881220235421809751254540594752243\\
52584907711670556013604839586446706324415722155397\\
53697817977846174064955149290862569321978468622482\\
83972241375657056057490261407972968652414535100474\\
82166370484403199890008895243450658541227588666881\\
16427171479924442928230863465674813919123162824586\\
17866458359124566529476545682848912883142607690042\\
24219022671055626321111109370544217506941658960408\\
07198403850962455444362981230987879927244284909188\\
84580156166097919133875499200524063689912560717606\\
05886116467109405077541002256983155200055935729725\\
71636269561882670428252483600823257530420752963450\\

Find the thirteen adjacent digits in the 1000-digit number that have the greatest product. What is the value of this product?


\section*{Answer}
The product is 23514624000 and the digits that got it are '5576689664895'
\section*{Idea} 
Pretty simple just substring all possible 13 long sub sections of the number. Multiply each digit in the string together return as PRODUCT. If the new product is greater then any product seen before overwrite the old one. 
\section*{Python code}
\begin{verbatim}
def prodOfDigits(numberString):
   product = 1
   for character in numberString:
      product = product * int(character)
   return product
def lrgstAdjacent(numberString, digits):
   largestProd = float('-inf')
   largestString = ""
   startIndex = 0
   while startIndex + digits <= len(numberString) - 1:
      newProd = prodOfDigits(numberString[startIndex : startIndex + digits])
      if newProd > largestProd:
         largestProd = newProd
         largestString = numberString[startIndex : startIndex + digits]

      startIndex = startIndex + 1
   return largestProd, largestString

input = "7316717653133062491922511967442657474235534919493496983520312774506326239578318016984801869478851843858615607891129494954595017379583319528532088055111254069874715852386305071569329096329522744304355766896648950445244523161731856403098711121722383113622298934233803081353362766142828064444866452387493035890729629049156044077239071381051585930796086670172427121883998797908792274921901699720888093776657273330010533678812202354218097512545405947522435258490771167055601360483958644670632441572215539753697817977846174064955149290862569321978468622482839722413756570560574902614079729686524145351004748216637048440319989000889524345065854122758866688116427171479924442928230863465674813919123162824586178664583591245665294765456828489128831426076900422421902267105562632111110937054421750694165896040807198403850962455444362981230987879927244284909188845801561660979191338754992005240636899125607176060588611646710940507754100225698315520005593572972571636269561882670428252483600823257530420752963450" 
print(lrgstAdjacent(input, 13))
\end{verbatim}
\end{document}