\documentclass{article}
\usepackage[margin=0.5in]{geometry}
\setlength\parindent{0pt}

\title{Euler's project problem 1}
\author{John Fox}
\Large
\begin{document}
\maketitle


\section*{Problem statement} 
The prime factors of 13195 are 5, 7, 13 and 29.

What is the largest prime factor of the number 600851475143 ?

\section*{Answer}
The largest prime factor of 600851475143 = 6857
\section*{Idea} Iterating up from factors starting at 2. If a number is divisible by the factor then we can divide the number by that factor before checking the next factor because the range [number/2 + 1, number] will not be divisible by anything. 

\section*{Python code}
\begin{verbatim}
def lrgstPrimeFac(num):
    list = []
    fac = 2
    while num > 1:
        while num % fac == 0:
            list.append(fac)
            num = num // fac
        fac = fac + 1
    return list
print(lrgstPrimeFac(600851475143))
\end{verbatim}
\end{document}